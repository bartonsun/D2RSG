\subsubsection{Дополнительные параметры шаблона}
\label{customParameters}
Автор шаблона может создать несколько параметров которые могут быть изменены пользователем по своему желанию перед генерацией сценария.\\
Каждый параметр описывается в виде списка, внутри списка \texttt{customParameters}, в одном их двух вариантов - диапазон или список значений.\\
\textbf{Основные параметры}
\begin{itemize}
\item \texttt{name} - отображаемое название параметра. Обязательное для заполнения.
\item \texttt{default} - значение параметра по умолчанию. Число. По умолчанию минимальное значение.
\end{itemize}
\textbf{Диапазон}
\begin{itemize}
\item \texttt{min} - минимальное значение параметра. Диапазон \texttt{[-9999 : 9999]}, 0 по умолчанию.
\item \texttt{max} - максимальное значение параметра. Диапазон \texttt{[-9999 : 9999]}, 0 по умолчанию.
\item \texttt{step} - шаг параметра. Диапазон \texttt{[1 : 9999]}, 1 по умолчанию.
\item \texttt{unit} - отображается рядом со значением, в качестве единицы измерения.
\end{itemize}
Пример:\\
Будет создан параметр "Сложность" с отображаемым значением по умолчанию 100\%, минимальным 50\%, максимальным 150\% и шагом 5\%.
\begin{figure}[H]
\lstinputlisting{docExamples/customParametersExample1.lua}
\end{figure}
\textbf{Список}
\begin{itemize}
\item \texttt{values} - список значений для выбора.
При этом \texttt{min} и \texttt{step} всегда равны 1. А \texttt{max} - количеству значений в \texttt{values}.
\item \texttt{values} - список возможных значений параметра.
\end{itemize}
Пример:\\
Будет создан параметр "Режим игры" с отображаемым значением по умолчанию FFA.
\begin{figure}[H]
\lstinputlisting{docExamples/customParametersExample2.lua}
\end{figure}
Выбранные значения передаются при генерации в виде списка в функцию \texttt{getTemplateContents} 
 третьим аргументом. Значения числовые, для ранее приведенных примеров будет передано \{100, 2\}, если выбраны значения по умолчанию.\\
Пример чтения параметров:
\begin{figure}[H]
\lstinputlisting{docExamples/customParametersExample3.lua}
\end{figure}